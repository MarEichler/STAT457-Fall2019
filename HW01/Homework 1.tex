\documentclass[12pt, letterpaper]{article}

\input{C:/Users/MarEichler17/Documents/MYE-Documents/SCHOOL/Northwestern/preamble}
 



 %line spacing
 \usepackage{setspace}
 \setstretch{1.2}
 
 %header, pagestyle
 \geometry{pass, letterpaper}
 \pagestyle{fancy}
 \fancyhf{} % sets both header and footer to nothing
 \renewcommand{\headrulewidth}{0pt}
 \lhead{STAT 457 Fall 2019 \\ Homework 01}
 \rhead{Martha Eichlersmith \\ Page \thepage \text{ of} \pageref{LastPage}}
 \setlength{\headsep}{48pt}
 \renewcommand{\sectionmark}[1]{\gdef\currsection{\thesection \ #1}}
 \renewcommand{\subsectionmark}[1]{\markright{\currsection\ $\mid$ \thesubsection \  #1}}



\begin{document}
1. Prove that $\Ex{ (X - \Ex{X})^2} = \Ex{X^2} - \left( \Ex{X} \right)^2$.  

\begin{align*}
\Ex{ \left( X- \Ex{X}^2 \right) } & = \Ex{ X^2 - 2 X \Ex{ X} + \Ex{ X}^2 }
\\
& = \Ex{ X^2} - 2 \Ex{X} \Ex{X} + \Ex{X}^2 
\\
& = \Ex{X^2} - 2 \Ex{X}^2 + \Ex{X}^2
\\
& = \Ex{X^2} - \Ex{X}^2
\end{align*} \vspace{-48pt} \\ \qed 
\vspace{12pt} 

2. Prove that if $X_1$ and $X_2$ are independent, then $\Ex{X_1 X_2} = \Ex{X_1} \Ex{X_2}$. 
 
\begin{align*}
\omit\rlap{\underline{Continuous}}
\\
\Ex{ X_1 X_2} & = \int_{-\infty}^\infty x_1 x_2 p_{X_1 X_2} (x_1, x_2) dx_1 dx_2
\\
& = \int_{-\infty}^\infty x_1 x_2 p_{X_1}(x_1) p_{X_2} (x_2) dx_1 dx_2 
&& X_1 \independent X_2 \implies p_{X_1 X_2} (x_1, x_2) = p_{X_1}(x_1) p_{X_2} (x_2)
\\
& = \int_{-\infty}^\infty x_1 p_{X_1}(x_1) d x_1
\int_{-\infty}^\infty x_2 p_{X_2}(x_2) d x_2
\\
&= \Ex{ X_1 } \Ex{X_2 }
\\[2ex]
\omit\rlap{\underline{Discrete}} 
\\
\Ex{X_1 X_2} & = \sum_{x_1} \sum_{x_2} x_1 x_2 p_{X_1, X_2} (x_1, x_2) 
\\
& = \sum_{x_1} \sum_{x_2} x_1 x_2 p_{X_1} (x_1) p_{X_2}  x_2) 
&&  X_1 \independent X_2 \implies p_{X_1 X_2} (x_1, x_2) = p_{X_1}(x_1) p_{X_2} (x_2)
\\
& = \left( \sum_{x_1} x_1 p_{X_1}(x_1) \right)
\left( \sum_{x_2} x_2 p_{X_2}(x_2) \right)
\end{align*} \vspace{-48pt} \\ \qed 
 
\newpage 
3. Prove that $\Var{X_1 + X_2} = \Var{X_1} + \Var{X_2} + 2\Cov{X_1, X_2}$, where $\Cov{X_1, X_2} = \Ex{ \left(X_1 - \Ex{X_1} \right) \left( X_2 - \Ex{ X_2} \right) } = \Ex{ X_1 X_2} - \Ex{X_1} \Ex{X_2}$. 

\begin{align*}
\Var{X_1 + X_2} &= \Ex{ (X_1 + X_2)^2} - \Ex{X_1 + X_2}^2 
\\
& = \Ex{ X_1^2 + 2 X_1 X_2 + X_2^2} - \left( \Ex{X_1} + \Ex{X_2} \right)^2 
\\
&= \Ex{X_1^2} + 2 \Ex{X_1 X_2} + \Ex{X_2^2} - \left( \Ex{X_1}^2 + 2 \Ex{X_1} \Ex{X_2} + \Ex{X_2}^2 \right)
\\
& = \Ex{X_1^2} + 2 \Ex{X_1 X_2} + \Ex{X_2^2} -  \Ex{X_1}^2 - 2 \Ex{X_1} \Ex{X_2} - \Ex{X_2}^2 
\\
& =
\underbrace{ \Ex{X_1^2} - \Ex{X_1}^2 }_{\Var{X_1}} 
+ 
\underbrace{ \Ex{X_2^2} - \Ex{X_2}^2}_{\Var{X_2}} 
+ 2 \cdot 
\underbrace{ \left( \Ex{ X_1 X_2} - \Ex{X_1} \Ex{X_2} \right) }_{ \Cov{X_1, X_2}}
\\
& = \Var{X_1} + \Var{X_2} + 2 \Cov{X_1, X_2}
\end{align*} \vspace{-48pt} \\ \qed 
\vspace{12pt} 

4. Prove or disprove via a counterexample: If $\Cor{X_1, X_2} = 0$ then $X_1$ and $X_2$ are independent 


$\Cor{X_1, X_2} = 0 \implies \Cov{X_1, X_2} = 0 \implies \Ex{ X_1 X_2} = \Ex{X_1}\Ex{X_2}$ \\
Need to find an example where $\Ex{ X_1 X_2} = \Ex{X_1}\Ex{X_2}$ but $X_1 \centernot\independent X_2$.   

Let $X_1$ be a random variable with possible values of $-1, \ 1$ where \\
$\pr(X_1 = -1) = \pr(X_2 = 1) = 0.5$ 

Let $X_2$ be a random variable with possible values of $-1, \ 0, \ 1$ \\
$X_2 = \begin{cases}
0 & \text{if } X_1=-1 \\
-1 \text{ or } 1 & \text{if } X_1 = 1 \text{ (both with probability 0.5)} 
\end{cases}$ 

Note that $X_2$ is dependent on $X_1$ 

$
\arraycolsep=1.4pt
\begin{array}{r c l } 
\Ex{X_1} &=& \sum_{x_1} x_1 p_{X_1}(x_1) 
\\
&=& -1 \cdot p_{X_1}(- 1) + 1 \cdot p_{X_1}(1)
\\
&=& -1 \cdot 0.5 + 1 \cdot 0.5 
\\
&=&
0 
\end{array}$ 
\hfill 
$
\arraycolsep=1.4pt
\begin{array}{r c l } 
\Ex{X_2} &=& \sum_{x_2} x_2 p_{X_2}(x_2) 
\\
&=& 0 \cdot 
\underbrace{p_{X_2}(0)}_{p_{X_1}(-1)}
 + -1 \cdot 
\underbrace{p_{X_2}(-1)}_{0.5 p_{X_1}(1)} + 1 \cdot 
\underbrace{ p_{X_2}( 1)} _{0.5  p_{X_1}(1)} 
\\
&=& 0 \cdot 0.5  -1 \cdot 0.25 + 1 \cdot 0.25 
\\
&=&
0 
\end{array}$ 

$
\arraycolsep=1.4pt
\begin{array}{r c l} 
\Ex{X_1 X_2} &=& \sum_{x_1} \sum_{x_1} x_1 x_2 p_{X_1 X_2}(x_1, x_2) 
\\
&=&  
-1 \cdot 0 \cdot \pr(X_1 = -1)
+
1 \cdot -1 \cdot \pr(X_1 = 1, X_2 = -1)
+ 
1 \cdot 1 \cdot \pr(X_1 = 1, X_2 = 1) 
\\
&=&
0 + \pr(X_1 = 1, X_2 = -1) - \pr(X_1 = 1, X_2 = 1)
\\
&& \begin{footnotesize} \text{Note: $\pr(X_1 = 1, X_2 = -1) = \pr(X_1 = 1, X_2 = 1)$} \end{footnotesize}
\\
&=& 0 
\end{array}$  

$\Ex{X_1} \Ex{X_2} = \Ex{X_1 X_2} \implies \Cor{X_1, X_2} = 0$ but $X_1 \centernot\independent X_2$. \qed 

\newpage 
5. Let $f$ and $g$ be non-decreasing on $\R$.  Assume that $\Ex{g^2(X)} < \infty$ and $\Ex{ f^2(X)} < \infty$.  Show $\Cov{ f(X), g(X)} \geq 0$.  Hint: Let $Y$ be independent of $X$, with the same distribution as $X$. Consider $\Ex{ \left\{ g(X) - g(Y) \right\} \left\{ f(X) - f(Y) \right\} }$. 

\begin{align*}
X \independent Y & \quad X, Y \text{ are from the same distribution}
\\
\omit\rlap{ Look at $\left\{ g(X) - g(Y) \right\} \left\{ f(X) - f(Y) \right\}$}
\\[1ex]
\omit\rlap{Option 1: $X < Y$}
\\
\left.
\arraycolsep=1.4pt
\begin{array}{r c l}
g(X) - g(Y) &\leq& 0 \\
f(X) - f(Y) &\leq& 0
\end{array}
\right\}
& \implies \left\{ g(X) - g(Y) \right\} \left\{ f(X) - f(Y) \right\} \geq 0 
\\[1ex]
\omit\rlap{Option 2: $X = Y$}
\\
\left.
\arraycolsep=1.4pt
\begin{array}{r c l}
g(X) - g(Y) &=& 0 \\
f(X) - f(Y) &=& 0
\end{array}
\right\}
& \implies \left\{ g(X) - g(Y) \right\} \left\{ f(X) - f(Y) \right\} = 0 
\\[1ex]
\omit\rlap{Option 3: $X > Y$}
\\
\left.
\arraycolsep=1.4pt
\begin{array}{r c l}
g(X) - g(Y) &\geq& 0 \\
f(X) - f(Y) &\geq& 0
\end{array}
\right\}
& \implies \left\{ g(X) - g(Y) \right\} \left\{ f(X) - f(Y) \right\} \geq 0 
\\[1ex]
\implies \left\{ g(X) - g(Y) \right\} \left\{ f(X) - f(Y) \right\} &\geq 0  
\\
\implies \Ex{ \left\{ g(X) - g(Y) \right\} \left\{ f(X) - f(Y) \right\} }
& \geq 0 
\\[2ex]
\omit\rlap{Let $g(Y) = \Ex{g(X)}$ and $f(Y) = \Ex{f(X)}$}
\\
\Ex{ \left\{ g(X) - \Ex{g(X)} \right\} \left\{ f(X) - \Ex{f(X)} \right\} }
& \geq 0 
\\
\Cov{ f(X), g(X)} & \geq 0 
\end{align*} \vspace{-48pt} \\ \qed 


\end{document}



%  \text{\textcolor{red}{$$}}
%  \text{\textcolor{blue}{$$}}
%  \text{\textcolor{Green}{$$}}